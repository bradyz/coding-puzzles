\documentclass[landscape]{article}
\usepackage[utf8]{inputenc}

\title{Random Math}
\date{Spring 2017}

\usepackage[landscape, top=1cm,bottom=1cm,left=1cm,right=1cm]{geometry}
\usepackage{amsthm}
\usepackage{amsmath}
\usepackage{amsfonts}
\usepackage{enumitem}
\usepackage[parfill]{parskip}

\setlength{\parskip}{6pt}
\pagenumbering{gobble}

\DeclareMathOperator*{\argmin}{arg\,min}

\theoremstyle{definition}
\newtheorem{theorem}{Theorem}[section]
\newtheorem{definition}{Definition}[section]

\begin{document}

\section{Physical Simulation}

\subsection{Rigid Body Time Integrator}

\begin{align*}
    L(c^i, \theta^i, c^{i+1}, \theta^{i+1})
        &= \Huge\sum_{bodies} \Big[ \frac{\rho V}{2} \Big\|\frac{c^{i+1} - c^i}{h} \Big\|^2 + \frac{\rho}{2} \omega(\theta^i, \theta^{i+1})^T R_{-\theta^i}^T M_I R_{-\theta^i} \omega(\theta^i, \theta^{i+1}) \Big] - V(c^{i+1}, \theta^{i+1}) \\ \\
    \frac{\partial}{\partial c^i} L(c^i, \theta^i, c^{i+1}, \theta^{i+1}) 
        &= \Huge\sum_{bodies} \Big[ -\rho V \frac{c^{i+1} - c^i}{h^2} \Big]^T \\ \\
    \frac{\partial}{\partial \theta^i} L(c^i, \theta^i, c^{i+1}, \theta^{i+1}) 
        &= \Huge\sum_{bodies} \frac{\partial}{\partial \theta^i} \Big[ \frac{\rho}{2} \omega(\theta^i, \theta^{i+1})^T R_{-\theta^i}^T M_I R_{-\theta^i} \omega(\theta^i, \theta^{i+1}) \Big] \\
        &= \Huge\sum_{bodies} \Big[ \rho \omega(\theta^i, \theta^{i+1})^T R_{-\theta^i}^T M_I \Big( \frac{\partial}{\partial \theta^i} \big[ R_{-\theta^i} \omega(\theta^i, \theta^{i+1}) \big] \Big) \Big] \\
        &= \Huge\sum_{bodies} \Big[ \rho \omega(\theta^i, \theta^{i+1})^T R_{-\theta^i}^T M_I \Big( \frac{\partial}{\partial \theta^i} R_{-\theta^i} \omega(\theta^i, \theta^{i+1}) + R_{-\theta^i} \frac{\partial}{\partial \theta^i} \omega(\theta^i, \theta^{i+1}) \Big) \Big] \\
        &= \Huge\sum_{bodies} \Big[ \rho \omega(\theta^i, \theta^{i+1})^T R_{-\theta^i}^T M_I \Big( -R_{-\theta^i} [\omega(\theta^i, \theta^{i+1})]_{\times} T(-\theta_i) + R_{-\theta^i} \frac{-1}{h} T(h \omega(\theta^i, \theta^{i+1}))^{-1} T(-\theta^i) \Big) \Big] \\
        &= \Huge\sum_{bodies} \Big[ \rho \omega(\theta^i, \theta^{i+1})^T R_{-\theta^i}^T M_I R_{-\theta^i} \Big(- [\omega(\theta^i, \theta^{i+1})]_{\times} T(-\theta_i) + \frac{-1}{h} T(h \omega(\theta^i, \theta^{i+1}))^{-1} \Big) T(-\theta^i) \Big] \\ \\
    \frac{\partial}{\partial c^{i+1}} L(c^i, \theta^i, c^{i+1}, \theta^{i+1})
        &= \Huge\sum_{bodies} \Big[ \rho V \frac{c^{i+1} - c^i}{h^2} \Big]^T - \frac{\partial}{\partial c^{i+1}} V(c^{i+1}, \theta^{i+1}) \\ \\
    \frac{\partial}{\partial \theta^{i+1}} L(c^i, \theta^i, c^{i+1}, \theta^{i+1}) 
        &= \Huge\sum_{bodies} \frac{\partial}{\partial \theta^{i+1}} \Big[ \frac{\rho}{2} \omega(\theta^i, \theta^{i+1})^T R_{-\theta^i}^T M_I R_{-\theta^i} \omega(\theta^i, \theta^{i+1}) \Big] - \frac{\partial}{\partial \theta^{i+1}} V(\theta^i, \theta^{i+1}) \\
        &= \Huge\sum_{bodies}  \Big[ \rho \omega(\theta^i, \theta^{i+1})^T R_{-\theta^i}^T M_I R_{-\theta^i} \big( \frac{\partial}{\partial \theta^{i+1}} \omega(\theta^i, \theta^{i+1}) \big) \Big] - \frac{\partial}{\partial \theta^{i+1}} V(\theta^i, \theta^{i+1}) \\
        &= \Huge\sum_{bodies}  \Big[ \rho \omega(\theta^i, \theta^{i+1})^T R_{-\theta^i}^T M_I R_{-\theta^i} \big( \frac{1}{h} T(-h\omega(\theta^i, \theta^{i+1}))^{-1} T(-\theta^{i+1}) \big) \Big] - \frac{\partial}{\partial \theta^{i+1}} V(\theta^i, \theta^{i+1}) \\
\end{align*}

\pagebreak
\subsection{Euler-Lagrange Equations}

\begin{align*}
    d_2 L(q^{i-1}, q^i) + d_1 L(q^i, q^{i+1}) &= 0 \\
    \intertext{For the tangential velocity, this is simply particle motion, and follows the force = mass velocity form.}
    \Huge\sum_{bodies} \Big[ \rho V \frac{c^i - c^{i-1}}{h^2} \Big]^T - \frac{\partial}{\partial c^i} V(c^i, \theta^i) + \Huge\sum_{bodies} \Big[ -\rho V \frac{c^{i+1} - c^i}{h^2} \Big]^T &= 0 \\
     \Huge\sum_{bodies} \Big[ \rho V \frac{c^{i+1} - c^i}{h^2} \Big]^T &= \Huge\sum_{bodies} \Big[ \rho V \frac{c^i - c^{i-1}}{h^2} \Big]^T - \frac{\partial}{\partial c^i} V(c^i, \theta^i) \\
    \Huge\sum_{bodies} \Big[ \rho V \dot{c}_{i+1} \Big] &= \Huge\sum_{bodies} \Big[ \rho V \dot{c}_i \Big] - h \frac{\partial}{\partial c^i} V(c^i, \theta^i)^T \\
    c_{i+1} &= c^i + h \dot{c}^i \\
    \dot{c}_{i+1} &= \dot{c}_i - h \frac{1}{\rho V} dV(c^i, \theta^i)^T \quad \forall i \in bodies
\end{align*}

Solving for angular velocity, we can see each rigid body is an independent system, and since the potential $V(c, \theta)$ in this case only includes gravity, $d_2 V(c, \theta) = 0$.
\begin{align*}
    \Big[ \rho \omega(\theta^{i-1}, \theta^{i})^T R_{-\theta^{i-1}}^T M_I R_{-\theta^{i-1}} (\frac{1}{h} T(-h \omega(\theta^{i-1}, \theta^{i}))^{-1} T(- \theta^{i}) ) \Big] + \Big[ \rho \omega(\theta^i, \theta^{i+1})^T R_{-\theta^i}^T M_I R_{-\theta^i} \Big( -[\omega(\theta^i, \theta^{i+1})]_{\times} T(-\theta_i) +  \frac{-1}{h} T(h \omega(\theta^i, \theta^{i+1}))^{-1} \Big) T(-\theta^i) \Big] &= 0 \\
    \Big[ \rho \omega(\theta^{i-1}, \theta^{i})^T R_{-\theta^{i-1}}^T M_I R_{-\theta^{i-1}} \frac{1}{h} T(-h \omega(\theta^{i-1}, \theta^{i}))^{-1} \Big] + \Big[ \rho \omega(\theta^i, \theta^{i+1})^T R_{-\theta^i}^T M_I R_{-\theta^i} \Big(- [\omega(\theta^i, \theta^{i+1})]_{\times} T(-\theta_i) + \frac{-1}{h} T(h \omega(\theta^i, \theta^{i+1}))^{-1} \Big) \Big] &= 0
\end{align*}
    
The goal is to compute the orientation at the next time step, $\theta^{i+1}$. This can be done through solving for $\omega(\theta^i, \theta^{i+1})$, since $R_{h \omega(\theta^i, \theta^{i+1})} = R_{\theta^{i+1}} R_{-\theta^i}$.

Using properties of rotation matrices, we can see $R_{-\theta^i}^T R_{h \omega(\theta^i, \theta^{i+1}} = R_{\theta^{i+1}}$.

Let $\omega^i = \omega(\theta^{i-1}, \theta^{i})$
\begin{align*}
    \Big[ \rho (\omega^i)^T R_{-\theta^{i-1}}^T M_I R_{-\theta^{i-1}} \frac{1}{h} T(-h \omega^i)^{-1} \Big] + \Big[ \rho (\omega^{i+1})^T R_{-\theta^i}^T M_I R_{-\theta^i} \Big( -[\omega^{i+1}]_{\times} T(-\theta_i) + \frac{-1}{h} T(h \omega^{i+1})^{-1} \Big) \Big] &= 0 \\
    \Big[ (\omega^i)^T R_{-\theta^{i-1}}^T M_I R_{-\theta^{i-1}} T(-h \omega^i)^{-1} \Big] + \Big[ (\omega^{i+1})^T R_{-\theta^i}^T M_I R_{-\theta^i} \Big( -h [\omega^{i+1}]_{\times} T(-\theta_i) + T(h \omega^{i+1})^{-1} \Big) \Big] &= 0
\end{align*}

We can use Newton's method to solve for $\omega^{i+1}$, where
\begin{align*}
    f(\omega^{i+1})
        &= \Big[ (\omega^i)^T R_{-\theta^{i-1}}^T M_I R_{-\theta^{i-1}} T(-h \omega^i)^{-1} \Big] + \Big[ (\omega^{i+1})^T R_{-\theta^i}^T M_I R_{-\theta^i} \Big( -h [\omega^{i+1}]_{\times} T(-\theta_i) + T(h \omega^{i+1})^{-1} \Big) \Big] \\
        &= 0 \\
    df(\omega^{i+1})
        &= R_{-\theta^i}^T M_I R_{-\theta^i} \Big( -h [\omega^{i+1}]_{\times} T(-\theta_i) + T(h \omega^{i+1})^{-1} \Big) + (\omega^i)^T R_{-\theta^{i-1}}^T M_I R_{-\theta^{i-1}} \Big( -h\ d[\omega^{i+1}]_{\times} T(-\theta_i) + h\ dT(h \omega^{i+1})^{-1} \Big) \\
        &\approx R_{-\theta^i}^T M_I R_{-\theta^i} \Big(T(h \omega^{i+1})^{-1} \Big) 
\end{align*}

\pagebreak

\subsection{Integration over a Triangle}

Given - arbitrary points of a triangle $u, v, w$, and a function $f$ that you want to integrate over the face of the triangle.

Let's consider the basic function $f(u, v, w) = 1$ for now, so this integration should give us the area of the triangle - $\| (v - u) \times (w - u) \|$.

First step is to parameterize this into a function of two parameters, $s, t$, using the basis vectors $v-u$, and $w-u$.

This can be thought of sliding across the plane that is covered by the triangle.

This leads us to the double integral -
\begin{align*}
\int_0^1 \int_0^{1-s} f(s, t)\ dt\ ds
    &= \int_0^1 t |_0^{1-s}\ ds \\
    &= \int_0^1 1 - s\ ds \\
    &= s - \frac{1}{2}s^2 |_0^1 \\
    &= \frac{1}{2}
\end{align*}

The true area of this triangle should be $\frac{\|(v - u) \times (w - u)\|}{2}$, so we can see we're off by a factor of $\|(v - u) \times (w - u)\|$, which is caused by fact that these basis vectors are no longer unit length.



\end{document}